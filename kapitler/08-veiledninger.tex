\chapter{Veiledninger}
\label{chap:veiledninger}
Dette kapittelet består av brukerveiledninger for implementering autentisering i webapplikasjon og android ved hjelp av Azure Active Directory.

\section{Brukerveiledning for web applikasjon}
\label{sec:veiledninger_brukerveiledningForWebApplikasjon}
Denne veiledningen viser hvordan en web applikasjon kan registreres i en AAD og hvordan AAD kan implementeres i en ASP.NET Web app laget i Visual Studio 2013. Autentisering er satt til "No Authentication".
\\
\\
Kodeeksemplene presentert i denne seksjonen er i hovedsak hentet fra et kodeeksempel i GitHub under AzureAD Samples som heter WebApp-OpenIDConnect-DotNet \cite{WebAppOpenIDConnectDotNet}. Microsoft referer til dette kodeeksemplet flere steder

\subsection{Registrering av web applikasjoner i AAD}
\label{subsec:veiledninger_brukerveiledningForWebApplikasjon_registreringAvWebApplikasjoneriAad}
For å kunne bruke AAD som autentiseringsverktøy må applikasjonene registreres i en AAD. Dette gjøres ved å logge inn i Azure portalen og navigere til aktuell AAD. Velg applikasjoner i topp menyen og trykk så på add i bunnmenyen. En dialogboks vil nå vises.
\\
\begin{enumerate}
\item Velg "Add an application my my organization is developing"
\\
\item Skriv inn navnet til applikasjonen. Dette blir navnet til applikasjonen i AAD. Velg web app/ web api og trykk på pilen for å gå videre
\\
\item Fyll inn sign-on url og App ID URI, se figur \ref{fig:registrereAppIAAD}. Sign-on url er base urlen til web applikasjonen eller web apiet. Her må https:// eller http:// være med i urlen. App ID URI er applikasjonen sin ID. Denne blir brukt når en bruker skal autentiseres for applikasjonen. App IDen blir sendt til AAD og forteller at det er denne applikasjonen brukeren ønsker token til. APP IDen er i tillegg inkludert i tokenet, slik at applikasjonen vet at det var til seg brukeren ville.
\pagebreak
\begin{figure}[!htbp]
\begin{center}
    {%
    \setlength{\fboxsep}{0pt}%
    \setlength{\fboxrule}{0.5pt}%
    \fbox{\includegraphics[scale=0.5]{graphics/veiledninger_implementereWebApp/regWebApp5}}%
    }%
    \caption{Registrere applikasjon i AAD}
    \label{fig:registrereAppIAAD}
    \end{center}
\end{figure}
\\
\item Klikk på ferdig. Nå er appikasjonen registrert i AAD og kan benyttes av brukere i AAD
\end{enumerate}


\subsection{Implementering av AAD i web applikasjon}
\label{subsec:veiledninger_brukerveiledningForWebApplikasjon_implementeringAvAADIWebApplikasjon}
For å kunne bruke AAD som autentiseringsverktøy for en web applikasjon må det legges til en del elementer i applikasjonen. Blandt annet må det legges inn noen biblioteker for å kunne benytte OpenID Connect protokollen og noe informasjon som AAD trenger for å autentisere brukere opp mot applikasjonen.

\subsection*{SSL}
SSL brukes for å sikre kommunikasjonen mellom applikasjonen og AAD. SSL kan aktiveres i Visual Studio. Se figur \ref{fig:aktiverSsl}
\\
\begin{enumerate}

  \item I solution explorer klikk på applikasjonen. da vises “properties” vinduet. Sett SSL-enabled til true og noter ned SSL URL.
    \begin{figure}[!htbp]
    \centering
        \begin{subfigure}{.3\textwidth}
            \centering
             {%
            \setlength{\fboxsep}{0pt}%
            \setlength{\fboxrule}{0.5pt}%
            \fbox{\includegraphics[scale=0.4]{graphics/veiledninger_implementereWebApp/impWebApp1_1}}%
            }%
            \label{fig:aktiverSsl1}
        \end{subfigure}%
        \begin{subfigure}{.5\textwidth}
            \centering
             {%
            \setlength{\fboxsep}{0pt}%
            \setlength{\fboxrule}{0.5pt}%
            \fbox{\includegraphics[scale=0.55]{graphics/veiledninger_implementereWebApp/impWebApp1_2}}%
            }%
            \label{fig:aktiverSsl2}
        \end{subfigure}
        \caption{Aktiver SSL}
        \label{fig:aktiverSsl}
    \end{figure}

  \item Høyreklikk på appikasjonen og velg properties. Velg web og endre “Project URL” til å SSL URL, se figur \ref{fig:aktiverSsl}.
  \pagebreak
    \begin{figure}[!htbp]
        \begin{center}
             {%
            \setlength{\fboxsep}{0pt}%
            \setlength{\fboxrule}{0.8pt}%
            \fbox{\includegraphics[scale=0.5]{graphics/veiledninger_implementereWebApp/impWebApp2}}%
            }%
            \caption{Registrer SSL url}
            \label{fig:registrerSslUrl}
        \end{center}
    \end{figure}
\end{enumerate}

\subsection*{Implementere OWIN pakker}

\begin{enumerate}
  \item For at AAD skal bruke OpenID Connect som autentiseringsprotokoll må det legges til noen biblioteker i applikasjonen. Åpne Nuget konsollen i Visual Studio og installer følgende Katana elementer, \ref{lst:katana_elementer}:
    \begin{lstlisting}[caption={Katanaelementer},label={lst:katana_elementer},numbers=left,escapeinside={@}{@}]
        Install-Package Microsoft.Owin.Security.OpenIdConnect -Pre
        Install-Package Microsoft.Owin.Security.Cookies -Pre
        Install-Package Microsoft.Owin.Host.SystemWeb -Pre
    \end{lstlisting}
\end{enumerate}

\subsection*{Implementere autentiseringslogikk}

\begin{enumerate}
\item Naviger til “App\_Start” mappen i prosjektet og legg til en C\# klasse som heter Startup.Auth.cs. Fjern App\_Start fra namespace og legg til referanser og kode i Startup.Auth.cs, se \ref{lst:startupAuth}. 
\begin{lstlisting}[captionpos=b, caption={Startup.Auth.cs},label={lst:startupAuth},numbers=left]
        using Owin;
using Microsoft.Owin.Security;
using Microsoft.Owin.Security.Cookies;
using Microsoft.Owin.Security.OpenIdConnect;
using System.Configuration;
using System.Globalization;
using System.Threading.Tasks;

namespace WebApp_OpenIDConnect_DotNet
{
    public partial class Startup
    {
        private static string clientId = 
            ConfigurationManager.AppSettings["ida:ClientId"];
        private static string aadInstance = 
            ConfigurationManager.AppSettings["ida:AADInstance"];
        private static string tenant = 
            ConfigurationManager.AppSettings["ida:Tenant"];
        private static string postLogoutRedirectUri
            ConfigurationManager.AppSettings["ida:PostLogoutRedirectUri"];

        string authority = 
        String.Format(CultureInfo.InvariantCulture, aadInstance, tenant);

        public void ConfigureAuth(IAppBuilder app)
        {
            app.SetDefaultSignInAsAuthenticationType(
            CookieAuthenticationDefaults.AuthenticationType);

            app.UseCookieAuthentication(new CookieAuthenticationOptions());

            app.UseOpenIdConnectAuthentication(
                new OpenIdConnectAuthenticationOptions
                {
                    ClientId = clientId,
                    Authority = authority,
                    PostLogoutRedirectUri = postLogoutRedirectUri,
                    Notifications = new OpenIdConnectAuthenticationNotifications
                    {
                        AuthenticationFailed = context => 
                        {
                            context.HandleResponse();
                            context.Response.Redirect("/Error?message="
                                + context.Exception.Message);
                            return Task.FromResult(0);
                        }
                    }
                });
        }
    }
}
\end{lstlisting}
\\
\item Høyreklikk på prosjektet, velg add, class og velg OWIN Startup Class. Kall klassen for Startup.cs Legg til kode, se \ref{lst:startup} i Startup.cs
\begin{lstlisting}[captionpos=b, caption={Startup.cs},label={lst:startup},numbers=left]
using System;
using System.Threading.Tasks;
using Microsoft.Owin;
using Owin;

[assembly: OwinStartup(typeof(WebApp_OpenIDConnect_DotNet.Startup))]

namespace WebApp_OpenIDConnect_DotNet
{    {
    public partial class Startup
    {
        public void Configuration(IAppBuilder app)
        {
            ConfigureAuth(app);
        }
    }
}
\end{lstlisting}
\\
\item Naviger til Views, Shared og lag en et nytt partial view som heter \_LoginPartial.cshtml. Legg til kode, ser \ref{lst:loginPartial} i \_LoginPartial.cshtml. denne koden vil legge til en logg inn og en logg ut knapp i applikasjonen.
\begin{lstlisting}[captionpos=b, caption={LoginPartial.cs},label={lst:loginPartial},numbers=left, language=HTML]
@if (Request.IsAuthenticated)
{
    <text>
        <ul class="nav navbar-nav navbar-right">
            <li class="navbar-text">
                Hello, @User.Identity.Name!
            </li>
            <li>
                @Html.ActionLink("Sign out", "SignOut", "Account")
            </li>
        </ul>
    </text>
}
else
{
    <ul class="nav navbar-nav navbar-right">
        <li>@Html.ActionLink(
        "Sign in", "SignIn", "Account", routeValues: null, 
        htmlAttributes: new { id = "loginLink" })</li>
    </ul>
}
\end{lstlisting}
\\
\item Naviger til Views, Shared og åpne \_Layout.cshtml. Legg til:
\begin{lstlisting}
    @Html.Partial("_LoginPartial").
\end{lstlisting}
Dette vil legge til innholdet i \_LoginPartial.cshtml i applikasjones design
\\
\item Legg til en ny controller, velg MVC 5 Controller Empty og kall den AccountController. Legg til kode i AccountController, se \ref{lst:accountController}.
\begin{lstlisting}[captionpos=b, caption={AccountController.cs},label={lst:accountController},numbers=left]
using System;
using System.Collections.Generic;
using System.Linq;
using System.Web;
using System.Web.Mvc;
using Microsoft.Owin.Security.Cookies;
using Microsoft.Owin.Security.OpenIdConnect;
using Microsoft.Owin.Security;

namespace WebApp_OpenIDConnect_DotNet.Controllers
{
    public class AccountController : Controller
    {
        public void SignIn()
        {
            // Send an OpenID Connect sign-in request.
            if (!Request.IsAuthenticated)
            {
                HttpContext.GetOwinContext()
                    .Authentication.Challenge(
                        new AuthenticationProperties { 
                            RedirectUri = "/" 
                        },
                        OpenIdConnectAuthenticationDefaults
                        .AuthenticationType
                    );
            }
        }
        public void SignOut()
        {
            // Send an OpenID Connect sign-out request.
            HttpContext.GetOwinContext()
                .Authentication.SignOut(
                    OpenIdConnectAuthenticationDefaults
                    .AuthenticationType,
                    CookieAuthenticationDefaults
                    .AuthenticationType
                );
        }
	}
}
\end{lstlisting}
\\
\item Åpne Web.config og legg til Azure AD nøkler i <appSettings>, se \ref{lst:webConfig}
\begin{lstlisting}[captionpos=b, caption={Web.config},label={lst:webConfig},numbers=left]
<add key="ida:ClientId" value="[Enter client ID from Azure Portal]" />
<add key="ida:Tenant" value="[Enter tenant name]" />
<add key="ida:AADInstance" value="https://login.microsoftonline.com/{0}" />
<add key="ida:PostLogoutRedirectUri" value="https://localhost:44320/" />
\end{lstlisting}
\\
    \begin{itemize}
    \item ClientId er en applikasjonsid som opprettes når applikasjoner registreres i en AAD. Denne kan finnes ved å gå inn på aktuell AAD, naviger til på applikasjoner og velg gjeldende applikasjon. Client ID finnes under Configure
    \item AADInstance vil si hvilken instanse av Azure som brukes. Her brukes https://login.windows.net/{0} som er den offentlige Azure AD instansen.
    \item Tenant er navnet til den aktuelle  AAD tenanten. Navnet finnes ved å navigere til AAD og velge domains. Her er det enten gitt et domene av Azure, eller detkan legges til ett eget domene.
    \item PostLogoutRedirectUri er url brukere vil bli viderekoblet til etter utlogging
    \end{itemize}
\end{enumerate}
\\
Nå er applikasjonen klar for å bruke Azure AD som autentiseringsverktøy.
Applikasjonen vil nå ha en logg inn knapp som vil viderekoble brukeren til Azure AD sin innloggingsportal. 
Hvis det er ønskelig at applikasjonen skal autentisere brukeren for å få tilgang til hele applikasjonen eller kun noen deler av applikasjoner kan [Authorize] attributten settes der det er ønskelig med autentisering. Det viser seg å være noen problemer rundt dette hvis brukeren ikke bruker https for å navigere til applikasjonen.
\\

\section{Android Brukerveiledning}
\label{sec:veiledninger_androidBrukerveliedning}
Lag en demoapp som bruker azure mobile service og azure aad for autentisering. Om man har en app allerede følger man tutorialen som angitt, men kopierer heller den nye koden inn i din eksisterende app. 
\\
\\
Brukerveiledningen har tatt utgangspunkt i følgende brukerveiledninger fra Microsoft:
\\\url{http://azure.microsoft.com/en-us/documentation/articles/mobile-services-android-get-started/}
\\\url{http://azure.microsoft.com/en-us/documentation/articles/mobile-services-android-get-started-users/}
\\\url{http://azure.microsoft.com/en-us/documentation/articles/mobile-services-how-to-register-active-directory-authentication/}
\\
\\
Du skal ikke trenge å lese noen andre toturials enn denne med mindre du mangler en Azure Active Directory og Android Studio installasjon.


\subsection{Del 1 - Lag en mobile service i azure og eksempel-app}
\label{subsec:veiledninger_androidBrukerveliedning_del1-LagEnMobileServiceIAzureogEksempel-app}
\begin{enumerate}
\item Logg inn i management portalen og opprett en ny mobile service.
\\
\begin{figure}[H]
    \centering
    \setlength{\fboxsep}{0pt}%
    \setlength{\fboxrule}{1pt}%
    \fbox{\includegraphics[scale=0.30]{graphics/veliedninger_implementeringAndroid/01-CreateMobileServices.png}}
    \caption{Creating a Mobile Service in Azure}
    %\label{fig:MobileServices}
\end{figure}

\\
\item Opprett navnet du vil at appen skal ha i AzureAD (URL), gjør gjerne dette så beskrivende som mulig.
\\
\item Velg type database du ønsker, her kan du bruke eksisterende SQL database, opprette ny SQL database eller opprette en gratis 20MB SQL database. Databasen må ligge i Azure eller være knyttet til Azure allerede om du skal bruke en eksisterende database.
\\
\begin{itemize}
\item “Region” sier hvor mobile servicen skal lagres, microsoft anbefaler å ha mobile service og database lagret i samme “region”.
\item “Backend” er for å legge inn backend API i tillegg til de spørringene en bruker allerede kan gjøre mot databasen.
\item Sett parametere for databasen. Skriv gjerne ned dette på et sikkert sted, kan hende det tar tid før man får bruk for denne informasjonen igjen. Gjentar at Microsoft anbefaler at databasen ligger i samme region som mobile-service, men dette er ikke et krav.
\end{itemize}
\bigskip

\item Etter at du har opprettet appen går du til Mobile Services i Azure portalen. Velg “Android”, deretter klikk på “create a new android app”
\\
\item Om du allerede har Android studio hoppe over dette punktet. Har du den ikke så last ned og installer android studio.
\\
\item Klikk på create table. Denne oppretter det du trenger for og kjøre eksempelappen. Last ned eksempelappen, det er denne appen denne tutorialen har tatt utgangspunkt i.
\\
\item Åpne appen i android studio, og la android studio kjøre bakgrunnsprosesser ferdig før du gjør noe. Du ser dette på nederste statuslinje i programmet. 
Om Android studio ikke vil godta sertifikatet som følger med prosjektet du har lastet ned så er dette kun knyttet til en https: kobling for og laste ned siste versjon av gradle. Denne pakken kan du hente over http kobling også, om du ønsker en rask løsning på dette.
\\
\begin{itemize}
\item Sertifikatfeilen lå i nedlastningen av siste gradle versjon. Rett opp dette ved å gå inn i filen gradle-wrapper.properties og endre distributionUrl fra https://... til http:// . Filen ligger i prosjektet under /gradle/wrapper/
\item Om Android studio kjører som det skal og ikke gir feilmeldinger kan du kjøre appen nå uten autentisering. Pakken du har lastet ned har automatisk lagt inn knytning til mobile service med uri og key. 
\item Om android sier du må laste ned noen sdk-er, følg bare anbefalte forslag fra android studio. 
\end{itemize}
\bigskip
\begin{figure}[H]
    \centering
    \setlength{\fboxsep}{0pt}%
    \setlength{\fboxrule}{1pt}%
    \fbox{\includegraphics[scale=0.30]{graphics/veliedninger_implementeringAndroid/01-URLchange.png}}
    \caption{Skjermdump fra gradle-wrapper.properties}
    %\label{fig:MobileServices}
\end{figure}

\\
\item Appen skal nå virke, selv om den er svært enkel. 
\\
\item Sjekk gjerne at du finner igjen oppgavene du lagrer i todolisten i databasetabellen i Azure. Appen og databasen er ikke brukerstyrt, så om du installerer appen på flere telefoner, vi alle telefonene få opp det samme.
\end{enumerate}

\subsection{Del 2 - Legg inn autentisering i applikasjonen.}
\label{subsec:veiledninger_androidBrukerveliedning_del2-LeggInnAutentiseringIApplikasjonen}
For og kunne bruke azure ad som identity provider i appen må dette registrerers både i app koden, i Azure Mobile Service i Azure og i Azure Active Directory. Denne toturialen forutsetter at du har registrert en app i Azure Mobile Service, slik som i del 1 av denne toturialen.
\\
\begin{enumerate}
\item Gå inn i Mobile Services og velg appen du ønsker å legge til identity provder for. 
\\
\item Velg identity og scroll ned listent til den identity provideren du ønsker å bruke. Kopier her url fra appen. 
\\
\begin{figure}[H]
    \centering
    \setlength{\fboxsep}{0pt}%
    \setlength{\fboxrule}{1pt}%
    \fbox{\includegraphics[scale=0.30]{graphics/veliedninger_implementeringAndroid/02-MobileServicesIdentity.png}}
    \caption{Kopier APP URL'en}
    %\label{fig:MobileServices}
\end{figure}
\\
\item Gå inn i den Azure Active Directory du ønsker skal ha tilgang og velg applications.
\\
\begin{itemize}
\item Klikk “ADD” nederst for og legge til en ny applikasjon.
\item Velg “Add an application my organization is developing”
\item Angi navnet applikasjonen skal ha i din Azure Active Directory
\item Velg “WEB APPLICATION AND/OR WEB API”
\item Klikk neste og angi den kopierte APP URL’n i begge boksene. 
\end{itemize}
\bigskip



\item Nå er applikasjonen lagt til i din Azure Active Directory, du skal nå hente ut “Client ID” for applikasjonen, for og knytte applikasjonen i Azure Active Directory sammen med Azure Mobile Services. 
\\
\begin{itemize}
\item Klikk på configure og scroll ned til “CLIENT ID”, kopier denne.
\item Gå tilbake til Azure Mobile Services, finn appen du kopierte URL’en fra tidligere, velg identity og scroll ned windows azure active directory, lim inn “CLIENT ID” i boksen under APP URL. 
\end{itemize}
\bigskip
\begin{figure}[H]
    \centering
    \setlength{\fboxsep}{0pt}%
    \setlength{\fboxrule}{1pt}%
    \fbox{\includegraphics[scale=0.30]{graphics/veliedninger_implementeringAndroid/02-App-properties.png}}
    \caption{Fyll inn URL i begge feltene}
    %\label{fig:MobileServices}
\end{figure}
\\
\item For at du skal kunne logge inn med domenet angitt i din Azure Active Directory må du hente ut dette domenet fra Azure Active Directory og lime det inn i Azure Mobile Services. 
\\
\begin{itemize}
\item Gå til din Azure Active Directory, klikk på “DOMAINS”. 
\item Om du ikke har opprette egne domener vil det være angitt et standard domene. Kopier de domenene du ønsker skal ha tilgang til appen. (kan fortsatt ikke logge inn uten godkjent bruker).
\item Gå til Azure Mobile Services, velg applikasjonen, velg Identity, scroll ned til Windows Azure Active Directory og lim inn domenene i “ALLOWED TENANTS”
\end{itemize}
\bigskip
\begin{figure}[H]
    \centering
    \setlength{\fboxsep}{0pt}%
    \setlength{\fboxrule}{1pt}%
    \fbox{\includegraphics[scale=0.30]{graphics/veliedninger_implementeringAndroid/02-MobileServicesIdentity.png}}
    \caption{Fyll inn app ID og domener}
    %\label{fig:MobileServices}
\end{figure}

\\
\item Vi skal nå hindre andre brukere enn de som er autentiserte og endre data i databasen. 
\begin{itemize}
\item Gi i Azure Mobile Services, velg applikasjonen du skal endre, velg “DATA” derretter “PREMISSIONS” 
\item Endre alle rettighetene til “Only Authenticated Users”
\item Husk å klikke “SAVE” nederst etter at du har endret.
\end{itemize}
\bigskip
\begin{figure}[H]
    \centering
    \setlength{\fboxsep}{0pt}%
    \setlength{\fboxrule}{1pt}%
    \fbox{\includegraphics[scale=0.30]{graphics/veliedninger_implementeringAndroid/02-App-premissions.png}}
    \caption{Endre alle "table premissions"}
    %\label{fig:MobileServices}
\end{figure}

\\
\item Vi skal nå legge autentisering inn i app koden i Android studio. Endringene vi gjør her refererer til filnavnet i eksempelprosjektet. Om du ikke bruker eksempelprosjektet legg dette inn i klassen hvor “public void onCreate()” ligger. 
Legg til følgende biblioteker for import.
\\
\begin{lstlisting}[numbers=left, captionpos=b,   caption={Eksempel som viser hvilke biblioteker som bør importeres.}, ]
    
import java.util.concurrent.ExecutionException;
import java.util.concurrent.atomic.AtomicBoolean;

import android.content.Context;
import android.content.SharedPreferences;
import android.content.SharedPreferences.Editor;

import com.microsoft.windowsazure.mobileservices.authentication.MobileServiceAuthenticationProvider;
import com.microsoft.windowsazure.mobileservices.authentication.MobileServiceUser;
\end{lstlisting}
\\
\item Derretter legg til “authenticate()” klassen i samme filen. Kopier inn eksempelkoden nedenfor. Legg merke til at denne koden kan avvike fra veiledninger på nettet, ettersom det her er spesifisert WindowsAzureActiveDirectory som identity provider. 
\\
\begin{lstlisting}[numbers=left, captionpos=b,   caption={Eksempel som viser hvordan authenticate klassen kan se ut.}, ]

private void authenticate() {
    // Login using the Azure AD provider

    ListenableFuture<MobileServiceUser> mLogin = 
    mClient.login(
        mobileServiceAuthenticationProvider.WindowsAzureActiveDirectory);
    
    Futures.addCallback(mLogin, 
        new FutureCallback<MobileServiceUser>() {
        
        @Override
        public void onFailure(Throwable exc) {         
            createAndShowDialog((Exception) exc, "Error");
        }           
        
        @Override
        public void onSuccess(MobileServiceUser user) {
            createAndShowDialog(String.format(
                "You are now logged in - %1$2s",
                user.getUserId()), "Success");
            createTable();  
        }
    });     
}

\end{lstlisting}
\\
\item Derretter skal vi legge til authenticate() klassen. Om du bruker et annet prosjekt enn eksempelprosjektet vil det trolig og legge dette inn etter at du har initert alle objekter men ikke begynt og kjøre noe logikk operasjoner. For demoprosjektet legger vi det inn rett etter at vi har initiert MobileServiceClient i onCreate(); klassen, derretter flytter vi de kodelinjene som lå under MobileServiceClient initieringen ned i en egen klasse som vi kaller createTable(); 
\\
\begin{lstlisting}[numbers=left, captionpos=b,   caption={Eksempel som viser hvordan onCreate() klassen ser ut etter at kode er flytte til en egen createTable() klasse.}, ]

try {
   // Create the Mobile Service Client instance, using the provided
   
   // Mobile Service URL and key
   mClient = new MobileServiceClient(
        "https://norkart-test1.azure-mobile.net/",
        "nSxoADeTLiHClFQShuomNyfOeSGAQL23",
        this).withFilter(new ProgressFilter());
   
   authenticate();
   
   } catch (MalformedURLException e) {
        createAndShowDialog(new Exception("There was an error creating 
        the Mobile Service. Verify the URL"), "Error");
}
\end{lstlisting}
\\
\begin{lstlisting}[numbers=left, captionpos=b,   caption={Eksempel som viser hvordan createTable() klassen kan se ut.}, ]

private void createTable() {

    // Get the Mobile Service Table instance to use
    mToDoTable = mClient.getTable(ToDoItem.class);

    mTextNewToDo = (EditText) findViewById(R.id.textNewToDo);

    // Create an adapter to bind the items with the view
    mAdapter = new ToDoItemAdapter(this, R.layout.row_list_to_do);
    ListView listViewToDo = (ListView) findViewById(R.id.listViewToDo);
    listViewToDo.setAdapter(mAdapter);

    // Load the items from the Mobile Service
    refreshItemsFromTable();
}
\end{lstlisting}
\\
\item Nå skal du kunne kjøre appen og måtte oppgi identity provider for og få logget inn. \\
\item Det er fortsatt mulig å lure seg unna autentiseringen, da vil applikasjonen krasje når den ikke får svaret den forventer fra databasen. Du kan enten legge inn en egen sjekk som sjekker om du er autentisert eller ikke. Alaternativt kan du be applikasjonen avslutte seg selv om brukeren skulle forsøke å omgå autentiseringen. Dette kan du gjøre ved å legge til en finish(); i callBack exception i authenticate() klassen slik som angitt nedenfor.
\\
\begin{lstlisting}[numbers=left, captionpos=b,   caption={Eksempel som viser hvordan onFailure() kan se ut om det legges til en finish(); .},]

Futures.addCallback(mLogin, new FutureCallback<MobileServiceUser>() {
  @Override
  public void onFailure(Throwable exc) {
      createAndShowDialog((Exception) exc, "Error");
      finish();
  }

  @Override
  public void onSuccess(MobileServiceUser user) {
      createAndShowDialog(String.format(
          "You are now logged in - %1$2s",
          user.getUserId()), "Success");
      createTable();
  }
});

\end{lstlisting}
\\
\item Nå skal du kunne logge inn i applikasjonen. Det er ikke enda lagt til noen logg-ut knapp. For å teste at du kan logge inn med ulike brukere må du fjerne appdata i innstillinger for applikasjonen i android systemet. 
\end{enumerate}

