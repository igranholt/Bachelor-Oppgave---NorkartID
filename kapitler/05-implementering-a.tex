\chapter{Konfigurasjon}
\label{chap:konfigurasjon}

\section{Registrering av brukere}
\label{sec:konfigurasjon_registreringAvBrukere}
{\color{red}TODO} \\
Koblinger mulig mot lokale databaser \\
Bruke Azure som en backup?\\
Organizational og Microsoft accounts

\subsection{Import av brukere}
\label{subsec:ikonfigurasjon_registreringAvBrukere_importAvBrukere}
{\color{red}TODO} \\
Import fra andre databaser \\
Etterimport av nye brukere fra lokal database \\
 
\subsection{Masse-registrering av brukere}
\label{subsec:konfigurasjon_registreringAvBrukere_masseRegistrering}
{\color{red}TODO} \\
Mulig å selvhåndtere igjennom epost?  \\
Sjekke hva rutiner og måter som dette anbefales å gjøre på \\

\subsection{Enkelt-registrering av brukere}
\label{subsec:konfigurasjon_registreringAvBrukere_enkeltRegistrering}
{\color{red}TODO} \\
Etter registrering og synkronisering relart med dette. \\

\section{Brukeradministrasjon}
\label{sec:konfigurasjon_grupperOgRoller}

\subsection{Single tenant work-around}
\label{subsec:konfigurasjon_grupperOgRoller_singleTenantWorkAround}
Om vår oppdragsgiver ønsker å ha alle sine brukere i en AD, men likevel ønsker at brukerene skal få kunne koble til med sine egne mailadresser, vil dette trolig være mulig i B2B og B2C løsningen som kommer.

\subsection{Grupper}
\label{subsec:konfigurasjon_grupperOgRoller_grupper}

\subsection{Roller}
\label{subsec:konfigurasjon_grupperOgRoller_roller}

\section{Egenadministrasjon}
\label{sec:konfigurasjon_endringAvBrukerprofil}
{\color{red}TODO} \\
Sjekke muligheter for selvadministrering av brukerprofil\\
Graph api vs myapps

\subsection{Endre passord}
\label{subsec:konfigurasjon_endringAvBrukerprofil_endrePassord}

\subsection*{Konfigurasjon i Azure}


\subsection*{Regsitrering av autentiseringsdata}


\section{Glemt passord}
\label{sec:konfigurasjon_glemtPassord}

\section{Legg til applikasjon i Azure AD}
\label{sec:konfigurasjon_leggTilApplikasjonIAzureAD}
{\color{red}TODO} \\

\section{Oppdatere applikasjoner i Azure AD}
\label{sec:konfigurasjon_leggTilApplikasjonIAzureAD_oppdatereApplikasjonIAzureAD}
{\color{red}TODO} \\

Etter applikasjonen er registrert i Azure AD er det mulig å oppdatere og legge til ekstra funksjonalitet knyttet til applikasjonen. Dette gjøres I Azure portalen. Det er mulig å oppdatere brukertilgang og adgang til andre applikasjoner. Det er også mulig å gjøre applikasjonen til en multi-tenant applikasjon, se seksjon \ref{subsec:konfigurasjon_grupperOgRoller_multiTenant}

\section{Autentiserings scenarier for AAD}
\label{sec:konfigurasjon_autentiseringsScenarierForAAD}
{\color{red}TODO, } \\
Hva skjer i bakgrunnen når AAD brukes for autentisering? 

\subsection{Webapplikasjon med OIDC}
\label{subsec:konfigurasjon_autentiseringsScenarierForAAD_webapplikasjonMedOIDC}
{\color{red}TODO, } \\
KJøre Fiddler tester og ha sekvensdiagram \\

\subsection{Mobilapplikasjon}
\label{subsec:konfigurasjon_autentiseringsScenarierForAAD_mobilApplikasjon}
{\color{red}TODO, } \\

\subsection{Desktop applikasjon}
\label{subsec:konfigurasjon_autentiseringsScenarierForAAD_desktopApplikasjon}
{\color{red}TODO, } \\

\section{Implementering mot de ulike platformene}
\label{sec:konfigurasjon_implementeringnMotDeUlikePlatformene}
Applikasjoner som skal bruke AAD som autentiseringsverktøy må implementere en del elementer. Denne seksjonen har til hensikt å forklare hvordan AAD kan implementeres i web- og nativeapplikasjoner  og er i hovedsak ment for utviklere i Norkart. I tillegg til disse forklaringene her er det laget veiledninger som viser hvordan dette kan gjøres steg for steg. Se vedlegg {\color{red}TODO} \\ 

\subsection{Webapplikasjon - ASP.NET MVC}
\label{subsec:konfigurasjon_implementeringMotDeUlikePlatformene_Webapplikasjon}

\subsection{Android-applikasjon}
\label{subsec:konfigurasjon_implementeringMotDeUlikePlatformene_leggetilNyApplikasjon_androidApplikasjon}
{\color{red}TODO} \\
Hvordan legge inn i Azure med tanke på autorisering og autentisering

\section{Lisensmodell}
\label{sec:konfigurasjon_lisensmodell}
{\color{red}TODO} \\